\documentclass{article}
\usepackage[UTF8]{ctex}
\usepackage{amsmath}
\title{热学作业}
\author{高冷的助教}
\begin{document}
\maketitle
\section*{前言}
里面有有意或无意的错误,抄答案者自重哦(⊙o⊙)
\section{2月15日}
\subsection*{1.4}
\subsubsection*{(1)}
"找出摄氏温度的气体压强为零的点"改成"找出该气体压强为零的点的摄氏温度"

两点$(0.01, 4.8*10^4), (100, 6.50*10^4)$
能回归到一个一次函数,令之为零,解得答案$-282.315^{\circ}C$

\subsubsection*{(2)}
$p$与$T$正比,故
\begin{equation}
p=6.5*10^4 \frac{0.01 + 273.15}{100 + 273.15}=4.75825*10^4 Pa
\end{equation}
\subsection*{1.5}
\begin{equation}
	\left\{\begin{aligned}
	t_s/t_{tr}&=1.36605\\
	t_s-t_{tr}&=23
	\end{aligned}\right.
\end{equation}
应该不用解释了,两个式子都是直接从题目中抄的,答案是
\begin{equation}
\left\{\begin{aligned}
t_s = 85.83\\
t_{tr}&=62.83
\end{aligned}\right.
\end{equation}
\subsection*{1.7}
\subsubsection*{(1)}
似乎意思是凑个$a,b,c$使得$\epsilon$在冰点沸点间正好均分$100$份.那就$\epsilon=a+t$了.
\subsubsection*{(2)}
\begin{equation}
	t-t_0 = T-T_0
\end{equation}
其中$T_0=0K$或者$273.15 ^{\circ}C$,则
\begin{equation}
	\epsilon = a+b(T-T_0)+c(T-T_0)^2
\end{equation}
"在绝对温度时"改成"在绝对零度时",则$T=0K$
\begin{equation}
\epsilon = a-b T_0 -cT_0^2 = a - 373.15 b +  273.15^2 c
\end{equation}
\subsubsection*{(3)}
注:使用摄氏温标,$t_0=0^{\circ}C$,且$t=0$时$\theta=200$
\begin{equation}
\left\{\begin{aligned}
\theta &\propto a+bt+ct^2 \\
200    &\propto a
\end{aligned}\right.
\end{equation}即
\begin{equation}
\frac{\theta}{200}=\frac{a+bt+ct^2}{a}
\end{equation}
代入$t=-100$
\begin{equation}
\theta = 200 - 20000b/a + 2000000 c/a
\end{equation}
\end{document}
